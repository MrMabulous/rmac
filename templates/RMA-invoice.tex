%%%%%%%%%%%%%  TITEL  %%%%%%%%%%%%%
% Art des Templates: Rechnungstemplate / invoice.tex
% Mandant: 
% Erstellungsdatum: 
% Version: 
% Ersteller: 
% � by Run my Accounts AG


%statische Elemente
\documentclass{scrartcl}
\usepackage[latin1]{inputenc}
\usepackage[ngerman]{babel}
\usepackage[T1]{fontenc}
\usepackage{tabularx}
\setlength{\parindent}{0pt}
\usepackage{graphicx}
\usepackage{ifthen}
\usepackage{mathptmx}
\font\ocr=ocrb10
\usepackage[absolute]{textpos}
\usepackage{colortbl}
\usepackage{color}
\usepackage{substr}


%dynamische Elemente
%Abst�nde Rand (Anpassung auf die PDF Hintergrund-Vorlage!)
\usepackage[a4paper,top=6cm,bottom=1cm,left=1.8cm,right=1.8cm]{geometry} 

%Schrift
\usepackage[scaled=.90]{helvet} 
\renewcommand{\familydefault}{\sfdefault}

%Tabellendefinitionen
\newcolumntype{L}[1]{>{\raggedright\let\newline\\\arraybackslash\hspace{0pt}}p{#1}} % linksb�ndig mit Breitenangabe
\newcolumntype{C}[1]{>{\centering\let\newline\\\arraybackslash\hspace{0pt}}p{#1}} % zentriert mit Breitenangabe
\newcolumntype{R}[1]{>{\raggedleft\let\newline\\\arraybackslash\hspace{0pt}}p{#1} }% rechtsb�ndig mit Breitenangabe 

%Hintergrund f�r erste Standard-Seite
\usepackage{eso-pic}
\newcommand\BackgroundPic{
\put(0,0){
\parbox[b][\paperheight]{\paperwidth}{%
\vfill
\includegraphics{../<%templates%>/hintergrund.pdf} 
}}}

%Hintergrund f�r Einzahlungsschein (ohne Footer)
\usepackage{eso-pic}
\newcommand\BackgroundPicEZ{
\put(0,0){
\parbox[b][\paperheight]{\paperwidth}{%
\vfill
\includegraphics{../<%templates%>/hintergrundez.pdf} 
}}}

\begin{document}
\AddToShipoutPicture{\BackgroundPic} % MUSS EINKOMMENTIERT WERDEN WENN HINTERGRUND
\shorthandoff{"}
\pagestyle{empty}

% Setzt die generelle Position der Seite. Der bedruckbare bereich kann horizontal / vertikal verschoben werden.
\textblockorigin{0.00cm}{0.00cm} % + nach links / + nach unten

%Schriftwahl
\normalfont

%%%%%%%%%%%%%  PAGEBREAK SECTION  %%%%%%%%%%%%%

<%pagebreak 54 15 24%>

%Die Artikeltabelle auf der aktuellen Seite wird damit abgeschlossen.
\end{tabularx} 

%Eine Zwischensummentabelle wird eingef�gt
\begin{tabularx}{\textwidth}{@{}Xr@{}} 
\hline
Zwischensumme &<%sumcarriedforward%> \\
\end{tabularx}

%Eine neue Seite wird initialisiert.
\newpage

%Die Artikeltabelle auf der neuen Seite wird initialisiert.
\begin{tabularx}{\textwidth}{@{}XR{1.4cm}R{1.4cm}R{2cm}R{2cm}@{}}

\textbf{Beschreibung} & \textbf{Menge} & \textbf{Einheit} & \textbf{Preis <%currency%>} & \textbf{Betrag <%currency%>} \\
\hline

%Der �bertrag der letzten Seite wird geschrieben.
\"{U}bertrag von Seite <%lastpage%> & & & & <%sumcarriedforward%> \\
<%end pagebreak%>

%%%%%%%%%%%%%  ADRESSKOPF  %%%%%%%%%%%%%

% Bei linker Positionierung des Adresskopfs / 2.25cm muss bei Nicht-Pingen Templates auf den Seitenrand unter \usepackage[a4paper, ... angepasst werden.
%\begin{textblock*}{6.5cm}(2.25cm,5.75cm)

% Bei rechter Positionierung des Adresskopfs:
\begin{textblock*}{6.5cm}(11.15cm,5.75cm)

\underline{\tiny{<%company%>, \BeforeSubString{ --}{<%address%>}, \BehindSubString{-- }{<%address%>}}}
\small
\ifthenelse{\equal{<%typeofcontact%>}{company}}{

%Firmenanschrift

<%name%>

\ifthenelse{\equal{<%contact%>}{}}{}{<%salutation%> <%firstname%> <%lastname%>}

}{

%Privatanschrift

<%salutation%>

<%firstname%> <%lastname%>

} 

\ifthenelse{\equal{address1}{}}{}{<%address1%>}

\ifthenelse{\equal{<%address2%>}{}}{}{<%address2%>}

<%zipcode%> <%city%>

\ifthenelse{\equal{<%country%>}{}}{}{<%country%>}

\end{textblock*}

\vspace{3.2cm}

%%%%%%%%%%%%%  RECHNUNGSKOPF  %%%%%%%%%%%%%

\parbox{\textwidth}{
\textbf{\large{Rechnung Nr. <%invnumber%>}\hfill {Datum:} <%invdate%>}
}

\vspace{0.5cm}

%Rechnungstitel
\ifthenelse{\equal{<%invdescription%>}{}}{}{
\textbf{<%invdescription%>}
\vspace{0.2cm}
}

%Bestellnummer
\ifthenelse{\equal{<%ordnumber%>}{}}{}{Bestellnummer: <%ordnumber%>
\vspace{0.2cm}
}

%Ihre Bestellnummer
\ifthenelse{\equal{<%ponumber%>}{}}{}{Ihre Bestellnummer: <%ponumber%>
\vspace{0.2cm}
}

%Text-Feld 1
\ifthenelse{\equal{<%shippingpoint%>}{}}{}{<%shippingpoint%>
\vspace{0.2cm}
}

%Text-Feld 2
\ifthenelse{\equal{<%shipvia%>}{}}{}{<%shipvia%>
\vspace{0.2cm}
}

%Text-Feld 3
\ifthenelse{\equal{<%waybill%>}{}}{}{<%waybill%>
\vspace{0.2cm}
}

\vspace{0.3cm}

%%%%%%%%%%%%%  ARTIKELTABELLE  %%%%%%%%%%%%%

\renewcommand{\arraystretch}{1.4} %Abstand zwischen den Zeilen

\begin{tabularx}{\textwidth}{@{}XR{1.4cm}R{1.4cm}R{2cm}R{2cm}@{}}

\textbf{Beschreibung} & \textbf{Menge} & \textbf{Einheit} & \textbf{Preis <%currency%>} & \textbf{Betrag <%currency%>} \\

\hline

<%foreach number%>
 <%description%> \ifthenelse{\equal{<%itemnotes%>}{}}{}{\newline <%itemnotes%>} 
\ifthenelse{\equal{<%deliverydate%>}{}}{}{\newline Lieferdatum: <%deliverydate%>} 
\ifthenelse{\equal{<%serialnumber%>}{}}{}{\newline Seriennummer: <%serialnumber%>}
\ifthenelse{\equal{<%projectnumber%>}{}}{}{\newline Projekt: <%projectnumber%>}
\ifthenelse{\equal{<%ordernumber%>}{}}{}{\newline Bestellnummer: <%ordernumber%>}
\ifthenelse{\equal{<%customerponumber%>}{}}{}{\newline Ihre Bestellnummer: <%customerponumber%>}
\ifthenelse{\equal{<%package%>}{}}{}{\newline Verpackung: <%package%>}
\ifthenelse{\equal{<%netweight%>}{}}{}{\newline Netto Gewicht: <%netweight%>}
\ifthenelse{\equal{<%grossweight%>}{}}{}{\newline Brutto Gewicht: <%grossweight%>}
& <%qty%> \ifthenelse{\equal{<%discountrate%>}{}}{}{\newline <%discountrate%> \%} & <%unit%> \ifthenelse{\equal{<%discountrate%>}{}}{}{\newline Rabatt}& <%sellprice%> \ifthenelse{\equal{<%discountrate%>}{}}{}{\newline <%netprice%>}  & <%linetotal%>\\
<%end number%>
\end{tabularx}

%%%%%%%%%%%%%  SUMMEN TEMPLATE  %%%%%%%%%%%%%

\ifthenelse{\equal{<%tax%>}{}}{

\parbox{\textwidth}{

\begin{tabularx}{\textwidth}{@{}Xr@{}}

\hline

<%if total%>
\textbf{Rechnungsbetrag <%currency%>\ifthenelse{\equal{<%tax%>}{}}{ (ohne MWST)} { inkl. MWST}}& \textbf{<%invtotal%>}\\
<%end total%>

\hline

\end{tabularx}

\vspace{0.2cm}

\footnotesize 

\begin{flushright}
\begin{tabular}{@{}lrrr@{}}
\textbf{MWST-Satz} & \textbf{Netto} &  \textbf{MWST} & \textbf{Total}\\
\hline
0 \%& <%invtotal%>  & 0.00 & <%invtotal%>\\

\end{tabular}
\end{flushright}

}

}{

\ifthenelse{\equal{<%taxincluded%>}{1}}{

\parbox{\textwidth}{

\begin{tabularx}{\textwidth}{@{}Xr@{}}

\hline

<%if total%>
\textbf{Rechnungsbetrag <%currency%>\ifthenelse{\equal{<%tax%>}{}}{} { inkl. MWST}}& \textbf{<%invtotal%>}\\
<%end total%>

\hline

\end{tabularx}

\vspace{0.2cm}

\footnotesize 

\begin{flushright}
\begin{tabular}{@{}lrrr@{}}
\textbf{MWST-Satz} & \textbf{Netto} &  \textbf{MWST} & \textbf{Total}\\
\hline
<%foreach tax%>
<%taxrate%> \%&<%taxbase%>  &<%tax%> & <%taxbaseinclusive%>\\
<%end tax%>
\end{tabular}

\end{flushright}
}

}{

\parbox{\textwidth}{
\begin{tabularx}{\textwidth}{@{}Xr@{}}

\hline

Zwischentotal <%currency%>& <%subtotal%> \\

<%foreach tax%>
MWST <%taxrate%> \% auf <%taxbase%> & <%tax%> \\
<%end tax%>

\hline

<%if total%>
\textbf{Rechnungsbetrag <%currency%>}& \textbf{<%invtotal%>}
<%end total%>

\end{tabularx}

\hline
}
}
}

%%%%%%%%%%%%%  RECHNUNGSFUSS  %%%%%%%%%%%%%

%Anpassung der Zeilenabst�nde
\renewcommand{\arraystretch}{1}

\vspace{0.5cm}

\parbox{\textwidth}{
Wir bedanken uns f�r Ihre \textbf{�berweisung innerhalb von <%terms%> Tagen} bis zum <%duedate%>.
}
\vspace{0.5cm}

\ifthenelse{\equal{<%notes%>}{}}{}{
<%notes%>
\vspace{0.5cm}
}

Freundliche Gr�sse

\vspace{0.5cm}

<%company%>

%%%%%%%%%%%%%  BEZAHLT TABELLE  %%%%%%%%%%%%%

%%%%%%%%%%%%%  FOOTER  %%%%%%%%%%%%%

%Die Seite wird aufgef�llt - der Footer ist ganz unten.
\vspace*{\fill}

\begin{footnotesize}
\hline

\vspace{0.4cm}

\parbox[t]{.34\textwidth}{
<%company%>\\
\BeforeSubString{ --}{<%address%>}\\
\BehindSubString{-- }{<%address%>}\\
\ifthenelse{\equal{<%2201_taxnumber%>}{}}{}{MWSt Nr.: <%2201_taxnumber%>}
}
\parbox[t]{.33\textwidth}{
\ifthenelse{\equal{<%tel%>}{}}{}{Tel.: <%tel%>\\}
\ifthenelse{\equal{<%fax%>}{}}{}{Fax: <%fax%>\\}
\ifthenelse{\equal{<%companyemail%>}{}}{}{E-Mail: <%companyemail%>\\}
\ifthenelse{\equal{<%companywebsite%>}{}}{}{Web: <%companywebsite%>\\}
}
\parbox[t]{.33\textwidth}{
\ifthenelse{\equal{<%bankname%>}{}}{}{Bank: <%bankname%>\\}
\ifthenelse{\equal{<%iban%>}{}}{}{IBAN: <%iban%>\\}
\ifthenelse{\equal{<%bic%>}{}}{}{BIC: <%bic%>\\}
}

\end{footnotesize}

%Abschluss des Dokuments, NUR falls kein Einzahlungsschein folgt.
%\end{document}

%%%%%%%%%%%%%  EZ UMBRUCH  %%%%%%%%%%%%%
\ifthenelse{\equal{<%rvc%>}{}}{

%Hier k�nnte man IBAN Infos anf�gen

}{

\ifthenelse{\equal{<%currency%>}{CHF}}{

\ifthenelse{\equal{<%invtotal%>}{0.00}}{

\end{document}

}{

\pagebreak

%\AddToShipoutPicture{\BackgroundPicEZ} % MUSS EINKOMMENTIERT WERDEN WENN HINTERGRUND

\pagestyle{empty}


%%%%%%%%%%%%%  EZ KOPF  %%%%%%%%%%%%%

\textbf{\large{Rechnung Nr. <%invnumber%>}}

\vspace{0.5cm}

\begin{tabular}{@{}lll}
\textbf{Rechnungsdatum:} & & <%invdate%>\\
\textbf{Rechnungsbetrag:} & & <%currency%> <%invtotal%>\\
\textbf{Zahlungsbedingungen:} & & <%terms%> Tage\\
\textbf{F�lligkeitsdatum:} & & <%duedate%>\\

\end{tabular}

%%%%%%%%%%%%%  EINZAHLUNGSSCHEIN  %%%%%%%%%%%%%

\ifthenelse{\equal{<%bankname%>}{POSTFINANCE}}{

%LINKS POSTFINANCE
%Einzahlung f�r
\begin{textblock*}{5.8cm}(0.6cm,19.6cm)
\begin{flushleft}
<%company%>

<%address%>
\end{flushleft}
\end{textblock*}

}{

%LINKS BANK
%Einzahlung f�r
\begin{textblock*}{5.8cm}(0.6cm,19.6cm)
\begin{flushleft}
<%bankname%>

<%bankzipcode%> <%bankcity%>
\end{flushleft}
\end{textblock*}
 
%Zugunsten von
\begin{textblock*}{5.8cm}(0.6cm,21.1cm)
\begin{flushleft}
<%company%>

\BeforeSubString{ --}{<%address%>}

\BehindSubString{-- }{<%address%>}

\end{flushleft}
\end{textblock*}

}

%Konto
\begin{textblock*}{3cm}(2.8cm,23.1cm)
\begin{flushleft}
\ocr{<%membernumber%>}

\end{flushleft}
\end{textblock*}

%Betrag Franken
\begin{textblock*}{3.7cm}(0.4cm,23.8cm)
\begin{flushright}
\ocr{<%integer_out_amount%>}

\end{flushright}
\end{textblock*}

%Betrag Rappen
\begin{textblock*}{1cm}(5.1cm,23.8cm)
\begin{flushleft}
\ocr{<%out_decimal%>}

\end{flushleft}
\end{textblock*}

%Einbezahlt von
\begin{textblock*}{5.5cm}(0.6cm,24.8cm)
\begin{flushleft}
\ocr{<%dcn group=5right%>}
\normalfont 

<%name%>

<%address1%>

<%zipcode%> <%city%>

\end{flushleft}
\end{textblock*}

\ifthenelse{\equal{<%bankname%>}{POSTFINANCE}}{

%MITTE POSTFINANCE
%Einzahlung f�r
\begin{textblock*}{6cm}(6.3cm,19.6cm)
\begin{flushleft}
<%company%>

<%address%>
\end{flushleft}
\end{textblock*}

}{

%MITTE BANK
%Einzahlung f�r 
\begin{textblock*}{6cm}(6.3cm,19.6cm)
\begin{flushleft}
<%bankname%>

<%bankzipcode%> <%bankcity%>
\end{flushleft}
\end{textblock*}
 
%Zugunsten von
\begin{textblock*}{6cm}(6.3cm,21.1cm)
\begin{flushleft}
<%company%>

\BeforeSubString{ --}{<%address%>}

\BehindSubString{-- }{<%address%>}
\end{flushleft}
\end{textblock*}

}

%Konto
\begin{textblock*}{6cm}(8.9cm,23.1cm)
\begin{flushleft}
\ocr{<%membernumber%>}

\end{flushleft}
\end{textblock*}

%Betrag Franken
\begin{textblock*}{3.9cm}(6.3cm,23.8cm)
\begin{flushright}
\ocr{<%integer_out_amount%>}
\end{flushright}
\end{textblock*}

%Betrag Rappen
\begin{textblock*}{1cm}(11.15cm,23.8cm)
\begin{flushleft}
\ocr{<%out_decimal%>}

\end{flushleft}
\end{textblock*}

%Kodierzeile
\begin{textblock*}{16cm}(6.85cm,27.2cm)
\begin{flushleft}
\ocr{<%rvc%>><%dcn%>+ <%bankstate%>>}
\end{flushleft}
\end{textblock*}

%RECHTS
%Referenz-Nr.
\begin{textblock*}{8.3cm}(12.38cm,22.1cm)
\begin{flushleft}
\ocr{<%dcn group=5right%>}

\end{flushleft}
\end{textblock*}

%Name
\begin{textblock*}{6.5cm}(12.38cm,24.4cm)
\begin{flushleft}
<%name%>

<%address1%>

<%zipcode%> <%city%>


\end{flushleft}
\end{textblock*}

%DIESE BEIDEN KLAMMERN WERDEN GEBRAUCHT, WENN DIE EINF�GUNG DES EZ AUTOMATISCH ERFOLGT
}
}
}
}

\end{document}