%%%%%%%%%%%%%  TITLE  %%%%%%%%%%%%%

% Typ: Kreditorenbuchung
% Name: ap_transaction.tex
% Sprache: English
% Erstellungsdatum: 21. M�rz 2016
% Mandant: 
% Version: 
% Ersteller: 
% � by Run my Accounts AG

%%%%%%%%%%%%%  DEFINITIONS  %%%%%%%%%%%%%
<%include definitions_packages.tex%>

<%include definitions_settings.tex%>

<%include content_footer.tex%>

%Der Footer wird auf der neuen Seite gesetzt (muss entfernt werden, wenn ein Hintergrund mit Footer verwendet wird)
\Footer

\newcommand{\AmountType}{invoice amount}

%%%%%%%%%%%%%  CONTENT  %%%%%%%%%%%%%

\vspace{1.5cm}

\centerline{\textbf{A C C O U N T S \ \  P A Y A B L E \ \  R E C E I P T}}

\vspace{1.5cm}

\parbox[t]{.5\textwidth}{
<%name%>

<%address1%>

<%address2%>

<%city%>
<%if state>
\hspace{-0.1cm}, <%state%>
<%end state%> <%zipcode%>

<%country%>

\vspace{0.3cm}

<%if contact%>
<%contact%>
\vspace{0.2cm}
<%end contact%>

<%if vendorphone%>
Tel: <%vendorphone%>
<%end vendorphone%>

<%if vendorfax%>
Fax: <%vendorfax%>
<%end vendorfax%>

<%email%>

<%if vendortaxnumber%>
VAT no.: <%vendortaxnumber%>
<%end vendortaxnumber%>
}
\hfill
\begin{tabular}[t]{ll}
  \textbf{Receipt no.:} & <%invnumber%> \\
  \textbf{Date:} & <%invdate%> \\
  \textbf{Date due:} & <%duedate%> \\
  <%if ponumber%>
    \textbf{PO no.:} & <%ponumber%> \\
  <%end ponumber%>
  <%if ordnumber%>
    \textbf{Order no.:} & <%ordnumber%> \\
  <%end ordnumber%>
  \textbf{Employee:} & <%employee%> \\
\end{tabular}

\vspace{1.5cm}

\renewcommand{\arraystretch}{1.3} %Abstand zwischen den Zeilen

\begin{tabularx}{\textwidth}{@{}L{1.9cm}L{4.3cm}L{5.07cm}L{1.3cm}R{3.15cm}@{}}

\hline
 \textbf{Account no.} & \textbf{Account} & \textbf{Description} &\textbf{Project} & \textbf{<%currency%> amount } \\
\hline
\end{tabularx}
\hline

<%foreach amount%>
\begin{tabularx}{\textwidth}{@{}L{1.9cm}L{4.3cm}L{5.07cm}L{1.3cm}R{3.15cm}@{}}
<%accno%> & <%account%> & <%description%> & <%projectnumber%> &<%amount%>\\
\end{tabularx}
\par
\begingroup
\rightskip=7.4cm % Parameter anpassen
\noindent <%itemnotes%>
\par
\endgroup
<%end amount%>

<%include content_sum_bookings.tex%>

\vspace{1cm}

\ifthenelse{\equal{<%notes%>}{}}{}{
<%notes%>
\vspace{1cm}
}


<%if paid_1%>
\begin{tabular}{@{}llllr@{}}
  \multicolumn{5}{c}{\textbf{Payments}} \\
  \hline
  \textbf{Date} & & \textbf{Receipt} & \textbf{Memo} & \textbf{<%currency%> amount} \\
<%end paid_1%>
<%foreach payment%>
  <%paymentdate%> & <%paymentaccount%> & <%paymentsource%> & <%paymentmemo%> & <%payment%> \\
<%end payment%>
<%if paid_1%>
\end{tabular}
<%end paid_1%>

%%%%%%%%%%%%%  END  %%%%%%%%%%%%%
 
\end{document}