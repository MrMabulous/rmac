
%%%%%%%%%%%%%  TITEL  %%%%%%%%%%%%%
% Art des Templates: Kreditorenlastschrift Buchung / debit_note.tex
% Mandant: 
% Erstellungsdatum: 
% Version: 
% Ersteller: 
% � by Run my Accounts AG


%statische Elemente
\documentclass{scrartcl}
\usepackage[latin1]{inputenc}
\usepackage[ngerman]{babel}
\usepackage[T1]{fontenc}
\usepackage{tabularx}
\setlength{\parindent}{0pt}
\usepackage{graphicx}
\usepackage{ifthen}
\usepackage{mathptmx}
\font\ocr=ocrb10
\usepackage[absolute]{textpos}
\usepackage{colortbl}
\usepackage{color}
\usepackage{substr}


%dynamische Elemente
%Abst�nde Rand (Anpassung auf die PDF Hintergrund-Vorlage!)
\usepackage[a4paper,top=6cm,bottom=1cm,left=1.8cm,right=1.8cm]{geometry} 

%Schrift
\usepackage[scaled=.90]{helvet} 
\renewcommand{\familydefault}{\sfdefault}

%Tabellendefinitionen
\newcolumntype{L}[1]{>{\raggedright\let\newline\\\arraybackslash\hspace{0pt}}p{#1}} % linksb�ndig mit Breitenangabe
\newcolumntype{C}[1]{>{\centering\let\newline\\\arraybackslash\hspace{0pt}}p{#1}} % zentriert mit Breitenangabe
\newcolumntype{R}[1]{>{\raggedleft\let\newline\\\arraybackslash\hspace{0pt}}p{#1} }% rechtsb�ndig mit Breitenangabe 

%Hintergrund f�r erste Standard-Seite
\usepackage{eso-pic}
\newcommand\BackgroundPic{
\put(0,0){
\parbox[b][\paperheight]{\paperwidth}{%
\vfill
\includegraphics{../<%templates%>/hintergrund.pdf} 
}}}

%Hintergrund f�r Einzahlungsschein (ohne Footer)
\usepackage{eso-pic}
\newcommand\BackgroundPicEZ{
\put(0,0){
\parbox[b][\paperheight]{\paperwidth}{%
\vfill
\includegraphics{../<%templates%>/hintergrundez.pdf} 
}}}

\begin{document}
\AddToShipoutPicture{\BackgroundPic} % MUSS EINKOMMENTIERT WERDEN WENN HINTERGRUND
\shorthandoff{"}
\pagestyle{empty}

% Setzt die generelle Position der Seite. Der bedruckbare bereich kann horizontal / vertikal verschoben werden.
\textblockorigin{0.00cm}{0.00cm} % + nach links / + nach unten

%Schriftwahl
\normalfont

\vspace{1.5cm}

\centerline{\textbf{K R E D I T O R E N L A S T S C H R I F T \hspace{0.3cm} B U C H U N G S B E L E G}}

\vspace{1.5cm}

\parbox[t]{.5\textwidth}{
<%name%>

<%address1%>

<%address2%>

<%city%>
<%if state>
\hspace{-0.1cm}, <%state%>
<%end state%> <%zipcode%>

<%country%>

\vspace{0.3cm}

<%if contact%>
<%contact%>
\vspace{0.2cm}
<%end contact%>

<%if vendorphone%>
Tel: <%vendorphone%>
<%end vendorphone%>

<%if vendorfax%>
Fax: <%vendorfax%>
<%end vendorfax%>

<%email%>

<%if vendortaxnumber%>
MWST Nr.: <%vendortaxnumber%>
<%end vendortaxnumber%>
}
\hfill
\begin{tabular}[t]{ll}
  \textbf{Beleg Nr.:} & <%invnumber%> \\
  \textbf{Datum:} & <%invdate%> \\
  \textbf{F�llgikeitsdatum:} & <%duedate%> \\
  <%if ponumber%>
    \textbf{PO-Nummer:} & <%ponumber%> \\
  <%end ponumber%>
  <%if ordnumber%>
    \textbf{Bestell-Nummer:} & <%ordnumber%> \\
  <%end ordnumber%>
  \textbf{Mitarbeiter:} & <%employee%> \\
\end{tabular}

\vspace{1.5cm}

%%%%%%%%%%%%%  ARTIKELTABELLE  %%%%%%%%%%%%%

\renewcommand{\arraystretch}{1.4} %Abstand zwischen den Zeilen

\begin{tabularx}{\textwidth}{@{}llXlr@{}}

 \textbf{Konto Nr.} & \textbf{Konto} & \textbf{Beschreibung} &\textbf{Projekt} & \textbf{Betrag <%currency%>} \\

\hline

<%foreach amount%>
<%accno%> & <%account%> & <%description%> & <%projectnumber%> &<%amount%>\\
<%end amount%>
\end{tabularx}

%%%%%%%%%%%%%  SUMMEN TEMPLATE  %%%%%%%%%%%%%

\ifthenelse{\equal{<%tax%>}{}}{

\parbox{\textwidth}{

\begin{tabularx}{\textwidth}{@{}Xr@{}}

\hline

<%if total%>
\textbf{Totalbetrag <%currency%>\ifthenelse{\equal{<%tax%>}{}}{ (ohne MWST)} { inkl. MWST}}& \textbf{<%invtotal%>}\\
<%end total%>

\hline

\end{tabularx}

\vspace{0.2cm}

\footnotesize 

\begin{flushright}
\begin{tabular}{@{}lrrr@{}}
\textbf{MWST-Satz} & \textbf{Netto} &  \textbf{MWST} & \textbf{Total}\\
\hline
0 \%& <%invtotal%>  & 0.00 & <%invtotal%>\\

\end{tabular}
\end{flushright}

}

}{

\ifthenelse{\equal{<%taxincluded%>}{1}}{

\parbox{\textwidth}{

\begin{tabularx}{\textwidth}{@{}Xr@{}}

\hline

<%if total%>
\textbf{Total <%currency%>\ifthenelse{\equal{<%tax%>}{}}{} { inkl. MWST}}& \textbf{<%invtotal%>}\\
<%end total%>

\hline

\end{tabularx}

\vspace{0.2cm}

\footnotesize 

\begin{flushright}
\begin{tabular}{@{}lr@{}}
\textbf{MWST-Satz} & \textbf{MWST}\\
\hline
<%foreach tax%>
<%taxrate%> \% &<%tax%>\\
<%end tax%>
\end{tabular}

\end{flushright}
}

}{

\parbox{\textwidth}{
\begin{tabularx}{\textwidth}{@{}Xr@{}}

\hline

Zwischentotal <%currency%>& <%subtotal%> \\

<%foreach tax%>
MWST <%taxrate%> \%  & <%tax%> \\
<%end tax%>

\hline

<%if total%>
\textbf{Total <%currency%>}& \textbf{<%invtotal%>}
<%end total%>

\end{tabularx}

\hline
}
}
}

%%%%%%%%%%%%%  FUSS  %%%%%%%%%%%%%

\vspace{1cm}

\ifthenelse{\equal{<%notes%>}{}}{}{
<%notes%>
\vspace{1cm}
}

%%%%%%%%%%%%%  ZAHLUNGS TABELLE %%%%%%%%%%%%%

<%if paid_1%>
\begin{tabular}{@{}llllr@{}}
  \multicolumn{5}{c}{\textbf{Zahlungen}} \\
  \hline
  \textbf{Datum} & & \textbf{Beleg} & \textbf{Notiz} & \textbf{Betrag <%currency%>} \\
<%end paid_1%>
<%foreach payment%>
  <%paymentdate%> & <%paymentaccount%> & <%paymentsource%> & <%paymentmemo%> & <%payment%> \\
<%end payment%>
<%if paid_1%>
\end{tabular}
<%end paid_1%>

%%%%%%%%%%%%%  FOOTER  %%%%%%%%%%%%%

%Die Seite wird aufgef�llt - der Footer ist ganz unten.
\vspace*{\fill}

\begin{footnotesize}
\hline

\vspace{0.4cm}

\parbox[t]{.34\textwidth}{
<%company%>\\
\BeforeSubString{ --}{<%address%>}\\
\BehindSubString{-- }{<%address%>}\\
\ifthenelse{\equal{<%2201_taxnumber%>}{}}{}{MWSt Nr.: <%2201_taxnumber%>}
}
\parbox[t]{.33\textwidth}{\hspace{1cm}
}
\parbox[t]{.33\textwidth}{
\ifthenelse{\equal{<%tel%>}{}}{}{Tel.: <%tel%>\\}
\ifthenelse{\equal{<%fax%>}{}}{}{Fax: <%fax%>\\}
\ifthenelse{\equal{<%companyemail%>}{}}{}{E-Mail: <%companyemail%>\\}
\ifthenelse{\equal{<%companywebsite%>}{}}{}{Web: <%companywebsite%>\\}
}

\end{footnotesize}

\end{document}