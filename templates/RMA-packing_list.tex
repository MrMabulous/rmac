%%%%%%%%%%%%%  TITEL  %%%%%%%%%%%%%
% Art des Templates: packing_list.tex
% Mandant: 
% Erstellungsdatum: 
% Version: 
% Ersteller: 
% � by Run my Accounts AG


%statische Elemente
\documentclass{scrartcl}
\usepackage[latin1]{inputenc}
\usepackage[ngerman]{babel}
\usepackage[T1]{fontenc}
\usepackage{tabularx}
\setlength{\parindent}{0pt}
\usepackage{graphicx}
\usepackage{ifthen}
\usepackage{mathptmx}
\font\ocr=ocrb10
\usepackage[absolute]{textpos}
\usepackage{colortbl}
\usepackage{color}
\usepackage{substr}


%dynamische Elemente
%Abst�nde Rand (Anpassung auf die PDF Hintergrund-Vorlage!)
\usepackage[a4paper,top=6cm,bottom=1cm,left=1.8cm,right=1.8cm]{geometry} 

%Schrift
\usepackage[scaled=.90]{helvet} 
\renewcommand{\familydefault}{\sfdefault}

%Tabellendefinitionen
\newcolumntype{L}[1]{>{\raggedright\let\newline\\\arraybackslash\hspace{0pt}}p{#1}} % linksb�ndig mit Breitenangabe
\newcolumntype{C}[1]{>{\centering\let\newline\\\arraybackslash\hspace{0pt}}p{#1}} % zentriert mit Breitenangabe
\newcolumntype{R}[1]{>{\raggedleft\let\newline\\\arraybackslash\hspace{0pt}}p{#1} }% rechtsb�ndig mit Breitenangabe 

%Hintergrund f�r erste Standard-Seite
\usepackage{eso-pic}
\newcommand\BackgroundPic{
\put(0,0){
\parbox[b][\paperheight]{\paperwidth}{%
\vfill
\includegraphics{../<%templates%>/hintergrund.pdf} 
}}}

%Hintergrund f�r Einzahlungsschein (ohne Footer)
\usepackage{eso-pic}
\newcommand\BackgroundPicEZ{
\put(0,0){
\parbox[b][\paperheight]{\paperwidth}{%
\vfill
\includegraphics{../<%templates%>/hintergrundez.pdf} 
}}}

\usepackage{hyperref}

\begin{document}

\AddToShipoutPicture{\BackgroundPic} % MUSS EINKOMMENTIERT WERDEN WENN HINTERGRUND
\shorthandoff{"}
\pagestyle{empty}

% Setzt die generelle Position der Seite. Der bedruckbare bereich kann horizontal / vertikal verschoben werden.
\textblockorigin{0.00cm}{0.00cm} % + nach links / + nach unten

%Schriftwahl
\normalfont

%%%%%%%%%%%%%  PAGEBREAK SECTION  %%%%%%%%%%%%%

<%pagebreak 68 15 24%>

%Die Artikeltabelle auf der aktuellen Seite wird damit abgeschlossen.
\end{tabularx} 

%Eine Zwischensummentabelle wird eingef�gt
\hline
\end{tabularx}

%Eine neue Seite wird initialisiert.
\newpage

%Die Artikeltabelle auf der neuen Seite wird initialisiert.
\begin{tabularx}{\textwidth}{@{}llXrr@{}}

\textbf{Pos.} & \textbf{Artikel} & \textbf{Beschreibung} & \textbf{Menge} & \textbf{Einheit} \\
\hline

<%end pagebreak%>

%%%%%%%%%%%%%  ADRESSKOPF  %%%%%%%%%%%%%

% Bei linker Positionierung des Adresskopfs / 2.25cm muss bei Nicht-Pingen Templates auf den Seitenrand unter \usepackage[a4paper, ... angepasst werden.
%\begin{textblock*}{6.5cm}(2.25cm,5.75cm)

% Bei rechter Positionierung des Adresskopfs:
\begin{textblock*}{6.5cm}(11.15cm,5.75cm)

\underline{\tiny{<%company%>, \BeforeSubString{ --}{<%address%>}, \BehindSubString{-- }{<%address%>}}}
\small

% Lieferanschrift

<%shiptoname%>

\ifthenelse{\equal{<%shiptocontact%>}{}}{}{<%shiptocontact%>}

\ifthenelse{\equal{<%shiptoaddress1%>}{}}{}{<%shiptoaddress1%>}

\ifthenelse{\equal{<%shiptoaddress2%>}{}}{}{<%shiptoaddress2%>}

<%shiptozipcode%> <%shiptocity%>

\ifthenelse{\equal{<%shiptocountry%>}{}}{}{<%shiptocountry%>}

\end{textblock*}

\vspace{3.2cm}

%%%%%%%%%%%%%  RECHNUNGSKOPF  %%%%%%%%%%%%%

\parbox{\textwidth}{
\textbf{\large{Lieferschein Nr. <%invnumber%>}\hfill {Datum:} <%invdate%>}
}

\vspace{0.5cm}

%Rechnungstitel
\ifthenelse{\equal{<%invdescription%>}{}}{}{
\textbf{<%invdescription%>}
\vspace{0.2cm}
}

%Bestellnummer
\ifthenelse{\equal{<%ordnumber%>}{}}{}{Bestellnummer: <%ordnumber%>
\vspace{0.2cm}
}

%Ihre Bestellnummer
\ifthenelse{\equal{<%ponumber%>}{}}{}{Ihre Bestellnummer: <%ponumber%>
\vspace{0.2cm}
}

%Text-Feld 1
\ifthenelse{\equal{<%shippingpoint%>}{}}{}{<%shippingpoint%>
\vspace{0.2cm}
}

%Text-Feld 2
\ifthenelse{\equal{<%shipvia%>}{}}{}{<%shipvia%>
\vspace{0.2cm}
}

%Text-Feld 3
\ifthenelse{\equal{<%waybill%>}{}}{}{<%waybill%>
\vspace{0.2cm}
}

\vspace{0.3cm}

%%%%%%%%%%%%%  ARTIKELTABELLE  %%%%%%%%%%%%%

\renewcommand{\arraystretch}{1.4} %Abstand zwischen den Zeilen

\begin{tabularx}{\textwidth}{@{}llXrr@{}}

\textbf{Pos.} & \textbf{Artikel} & \textbf{Beschreibung} & \textbf{Menge} & \textbf{Einheit} \\

\hline

<%foreach number%>
 <%runningnumber%> & <%number%>& <%description%> \ifthenelse{\equal{<%itemnotes%>}{}}{}{\newline <%itemnotes%>} 
\ifthenelse{\equal{<%deliverydate%>}{}}{}{\newline Lieferdatum: <%deliverydate%>} 
\ifthenelse{\equal{<%serialnumber%>}{}}{}{\newline Seriennummer: <%serialnumber%>}
\ifthenelse{\equal{<%projectnumber%>}{}}{}{\newline Projekt: <%projectnumber%>}
\ifthenelse{\equal{<%ordernumber%>}{}}{}{\newline Bestellnummer: <%ordernumber%>}
\ifthenelse{\equal{<%customerponumber%>}{}}{}{\newline Ihre Bestellnummer: <%customerponumber%>}
\ifthenelse{\equal{<%package%>}{}}{}{\newline Verpackung: <%package%>}
\ifthenelse{\equal{<%netweight%>}{}}{}{\newline Netto Gewicht: <%netweight%>}
\ifthenelse{\equal{<%grossweight%>}{}}{}{\newline Brutto Gewicht: <%grossweight%>}
& <%qty%> & <%unit%>\\
<%end number%>
\end{tabularx}
\hline

%%%%%%%%%%%%%  RECHNUNGSFUSS  %%%%%%%%%%%%%

%Anpassung der Zeilenabst�nde
\renewcommand{\arraystretch}{1}

\vspace{0.5cm}

\ifthenelse{\equal{<%notes%>}{}}{}{
<%notes%>
\vspace{0.5cm}
}

%%%%%%%%%%%%%  BEZAHLT TABELLE  %%%%%%%%%%%%%

%%%%%%%%%%%%%  FOOTER  %%%%%%%%%%%%%

%Die Seite wird aufgef�llt - der Footer ist ganz unten.
\vspace*{\fill}

\begin{footnotesize}
\hline

\vspace{0.4cm}

\parbox[t]{.34\textwidth}{
<%company%>\\
\BeforeSubString{ --}{<%address%>}\\
\BehindSubString{-- }{<%address%>}\\
}
\parbox[t]{.33\textwidth}{\hspace{1cm}
}
\parbox[t]{.33\textwidth}{
\ifthenelse{\equal{<%tel%>}{}}{}{Tel.: <%tel%>\\}
\ifthenelse{\equal{<%fax%>}{}}{}{Fax: <%fax%>\\}
\ifthenelse{\equal{<%companyemail%>}{}}{}{E-Mail: <%companyemail%>\\}
\ifthenelse{\equal{<%companywebsite%>}{}}{}{Web: <%companywebsite%>\\}
}

\end{footnotesize}

%Abschluss des Dokuments, NUR falls kein Einzahlungsschein folgt.
\end{document}